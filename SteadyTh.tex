\UseRawInputEncoding

\documentclass[12pt]{article}

\usepackage{amsmath}
\usepackage{systeme}
\usepackage[utf8]{inputenc}
%\usepackage{ragged2e}
\usepackage[english,ukrainian]{babel}

%\title{Title}
\title{Розв\textquotesingle язок задачі №5.26}
\author{Косицький, Озернюк, Панченко}

\begin{document}
\maketitle

\section*{Умова}

Вирішити найпростішу задачу класичного варіаційного числення

\begin{align*}
  & \int_{2}^{3} (x^2 - 1)\cdot (y')^2 \,dx \rightarrow extr; & y(2) = 0 \text{        і        } y(3) = 1.
\end{align*}

\section*{Розв\textquotesingle язок}

Складемо та розв\textquotesingle яжемо рівняння Ейлера

\begin{align*}
  & \frac{\partial F}{\partial y} - \frac{d}{dx}\frac{\partial F}{\partial y'} = 0,
\end{align*}
 де $F(x, y, y') = (x^2 - 1)\cdot (y')^2$ - підінтегральний вираз.

Розв\textquotesingle язавши рівняння Ейлера, ми знайдемо сімейство екстремалей, серед яких і будемо шукати екстремуми нашого функціоналу.
Оскільки $F$ явно від $y$ не залежить, то

\begin{align*}
  & \frac{\partial F}{\partial y} = 0.
\end{align*}
Розглядаючи похідну $F$ по $y'$ помічаємо, що вираз $(x^2 - 1)$ - константа, а тому

\begin{align*}
  & \frac{\partial F}{\partial y'} = 2(x^2 - 1)\cdot y'.
\end{align*}

Продиференціювавши останній вираз по змінній $x$, маємо

\begin{align*}
  & \frac{d}{dx}\frac{\partial F}{\partial y'} = 2(2xy' + y''(x^2 - 1)).
\end{align*}

Тепер, знайшовши усе необхідне, можемо скласти рівняння Ейлера

\begin{align*}
  & 2xy' + y''(x^2 - 1) = 0.
\end{align*}

Розв\textquotesingle яжемо його. Для цього помітимо, що ліва частина - це похідна від $y'(x^2 - 1)$. Тобто

\begin{align*}
  & 2xy' + y''(x^2 - 1) = 0 \iff \\
  & (y'(x^2 - 1))' = 0 \iff \\
  & y'(x^2 - 1) = c_1 \iff \\
  & \, dy = \frac{1}{x^2 - 1}\cdot c_1 \, dx = (\frac{1}{x - 1} - \frac{1}{x + 1})\cdot \frac{c_1}{2} \, dx.
\end{align*}

Проінтегруємо ліву та праву частини. Отримаємо

\begin{align*}
  & \int \, dy = \int (\frac{1}{x - 1} - \frac{1}{x + 1})\cdot \frac{c_1}{2} \, dx \iff \\
  & y = \frac{c_1}{2}\cdot ln(\frac{x - 1}{x + 1}) + c_2.
\end{align*}

Відсутність модуля у логарифмі пояснюється довільністю константи $c_1$.

Отож, наразі ми отримали сімейство екстремалей. Знайдемо саме той розв\textquotesingle язок рівняння Ейлера, що задовільняє нашим початковим умовам. Для цього підставимо значення $x = 2$ та $x = 3$, а потім розв\textquotesingle яжемо систему, невідомими якої є константи $c_1$ і $c_2$. Система матиме вигляд

\begin{equation*}
  \begin{cases}
    0 = \frac{c_1}{2}\cdot ln(\frac{1}{3}) + c_2, \\
    1 = \frac{c_1}{2}\cdot ln(\frac{1}{2}) + c_2
  \end{cases} \iff \\
  \begin{cases}
    \frac{c_1}{2}\cdot ln(3) = c_2, \\
    1 = \frac{c_1}{2}\cdot ln(\frac{1}{2}) + c_2
  \end{cases}
\end{equation*}

Підставивши значення $c_2$ з першого рівняння в друге, отримаємо

\begin{align*}
  & 1 = \frac{c_1}{2}\cdot ln(\frac{1}{2}) + \frac{c_1}{2}\cdot ln(3) \iff \\
  & \frac{1}{ln(\frac{1}{2}) + ln(3)} = \frac{c_1}{2} \iff c_1 = \frac{2}{ln(\frac{3}{2}).}
\end{align*}

Значення $c_2$ знайдемо з першого рівняння. Тобто

\begin{align*}
  & c_2 = \frac{c_1}{2}\cdot ln(3) = \frac{ln(3)}{ln(\frac{3}{2})}.
\end{align*}

Отож, на даний момент ми маємо лише одну єдину допустиму екстремаль

\begin{align*}
  &\tilde{y}(x) = \frac{1}{ln(\frac{3}{2})}\cdot ln(\frac{x - 1}{x + 1}) + \frac{ln(3)}{ln(\frac{3}{2})} = \frac{ln(3 \cdot \frac{x - 1}{x + 1})}{ln(\frac{3}{2})}.
\end{align*}

Дослідімо на екстремум. Розглянемо довільну функцію $h \in C^{1}_{[2, 3]}$, для якої $h(2) = h(3) = 0$. Якщо $J(\tilde{y} + h) - J(\tilde{y}) \geq 0$ для будь-якої функції $h$, то досягається слабкий глобальний мінімум. Якщо виконується нерівність в інший бік, то досягається слабкий глобальний максимум. Якщо нерівність виконується лише в деякому околі $\tilde{y}$, то говорять про локальний екстремум. Якщо ж в будь-якому околі існують такі функції $h$, що задовільняють обом нерівностям, то екстремум не досягається. Дослідимо різницю

\begin{align*}
  J(\tilde{y} + h) - J(\tilde{y}) & = \int_{2}^{3}(x^2 - 1)(\tilde{y}' + h')^2 \,dx - \int_{2}^{3}(x^2 - 1)(\tilde{y}')^2 \,dx = \\
  & = \int_{2}^{3} ((x^2 - 1)\tilde{y}'^2 + 2\tilde{y}'h'(x^2 - 1) + (x^2 - 1)h'^2 - (x^2 - 1)\tilde{y}'^2) \, dx = \\
  & = \int_{2}^{3} (2\tilde{y}'h'(x^2 - 1) + (x^2 - 1)h'^2) \, dx.
\end{align*}

Підставимо значення $\tilde{y}'$ у підінтегральний вираз. Оскільки $\tilde{y}' = \frac{c_1}{x^2 - 1} = \frac{2}{ln(\frac{3}{2})\cdot (x^2 - 1)}$, то

\begin{align*}
  J(\tilde{y} + h) - J(\tilde{y}) & = \int_{2}^{3} \frac{4}{ln(\frac{3}{2})}h' \, dx + \int_{2}^{3} (x^2 - 1)h'^2 \, dx = \\
  & = \frac{4}{ln(\frac{3}{2})} \int_{2}^{3} h' \, dx + \int_{2}^{3} (x^2 - 1)h'^2 \, dx = \\
  & = \frac{4}{ln(\frac{3}{2})} h(x) \Big|_{2}^{3} + \int_{2}^{3} (x^2 - 1)h'^2 \, dx = \\
  & = \int_{2}^{3} (x^2 - 1)h'^2 \, dx \geq 0.
\end{align*}

Отже, наша різниця функціоналів для будь-якої функції $h$ завжди невід\textquotesingle ємна. Це означає, що \textbf{функціонал} в нашій допустимій екстремалі $\tilde{y}$ \textbf{досягає глобального мінімуму}.

\end{document}
